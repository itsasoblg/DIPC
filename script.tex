\begin{document}
\documentclass{report}

SLIDE 1

In solid-state physics, the Tight Binding Model is an approximation to the calculation of the band structure of different materials. It's based on the fact that we can model a system of multiple atoms as the superposition of their wavefunctions. 

It is particularly useful for material where the electrons are close and bound to the nucleus. 

We can construct the Hamiltonian of this system, which can be expressed as the Hamiltonian associated with the atomic potential plus the perturbation to get the true potential of the system.

SLIDE 2

The wavefunction of the system is a combination of the individual atomic wavefunctions. Since this wavefunction corresponds to a infinitely periodic system it has to satisfy Bloch's Theorem.

SLIDE 3 

The most basic and simple example that can be done with the Tight Binding Model is a 1D chain where each atom only interacts with their nearest neighbours. Therefore, we can define an atomic TISE and a total one. Using the characteristic equation, we can diagonalize the matrix and obtain the eigenvalues, which take the form of trigonometric functions.

SLIDE 4

Utilizing these equations, we can model the electronic band structure of the system, as seen in the slide. 

SLIDE 5

In this case, instead of 2 nearest neighbours, we can see that each atom in our system will have 4 nearest neighbours, which means that we will have to include 2x2 matrices in our equations.
\\
The $\alpha$ elements in the matrix are the onsite energies of the atoms, which can be set to 0.

SLIDE 6 

As we can see, a band gap opens in the energy bands of this new system. 



 
\end{document}